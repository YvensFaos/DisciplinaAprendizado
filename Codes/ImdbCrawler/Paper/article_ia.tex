\documentclass[twoside]{article}
\usepackage{lipsum} % Package to generate dummy text throughout this template
\usepackage[utf8]{inputenc}
\usepackage[sc]{mathpazo} % Use the Palatino font
%\usepackage[T1]{fontenc} % Use 8-bit encoding that has 256 glyphs
\linespread{1.05} % Line spacing - Palatino needs more space between lines
\usepackage{microtype} % Slightly tweak font spacing for aesthetics
\usepackage[hmarginratio=1:1,top=32mm,columnsep=20pt]{geometry} % Document margins
\usepackage{multicol} % Used for the two-column layout of the document
\usepackage[hang, small,labelfont=bf,up,textfont=it,up]{caption} % Custom captions under/above floats in tables or figures
\usepackage{booktabs} % Horizontal rules in tables
\usepackage{float} % Required for tables and figures in the multi-column environment - they need to be placed in specific locations with the [H] (e.g. \begin{table}[H])
\usepackage{hyperref} % For hyperlinks in the PDF
\usepackage{lettrine} % The lettrine is the first enlarged letter at the beginning of the text
\usepackage{paralist} % Used for the compactitem environment which makes bullet points with less space between them
\usepackage{abstract} % Allows abstract customization
\renewcommand{\abstractnamefont}{\normalfont\bfseries} % Set the "Abstract" text to bold
\renewcommand{\abstracttextfont}{\normalfont\small\itshape} % Set the abstract itself to small italic text
\usepackage{titlesec} % Allows customization of titles
\renewcommand\thesection{\Roman{section}} % Roman numerals for the sections
\renewcommand\thesubsection{\Roman{subsection}} % Roman numerals for subsections
\titleformat{\section}[block]{\large\scshape\centering}{\thesection.}{1em}{} % Change the look of the section titles
\titleformat{\subsection}[block]{\large}{\thesubsection.}{1em}{} % Change the look of the section titles
\usepackage{fancyhdr} % Headers and footers
\pagestyle{fancy} % All pages have headers and footers
\fancyhead{} % Blank out the default header
\fancyfoot{} % Blank out the default footer
\fancyhead[C]{Artigo aí de Machine Learning title $\bullet$ Dezembro 2014 $\bullet$ } % Custom header text
\fancyfoot[RO,LE]{\thepage} % Custom footer text

%----------------------------------------------------------------------------------------
%	TITLE SECTION
%----------------------------------------------------------------------------------------

\title{\vspace{-15mm}\fontsize{24pt}{10pt}\selectfont\textbf{Artigo aí de Machine Learning}} % Article title

\author{
\large
\textsc{Yvens Rebouças}\\[2mm] % Your name
\normalsize Universidade de Fortaleza \\ % Your institution
\normalsize \href{mailto:yvensre@gmail.com}{yvensre@gmail.com} % Your email address
\vspace{-5mm}
}
\date{}

%----------------------------------------------------------------------------------------

\begin{document}

\maketitle % Insert title

\thispagestyle{fancy} % All pages have headers and footers

%----------------------------------------------------------------------------------------
%	ARTICLE CONTENTS
%----------------------------------------------------------------------------------------

\begin{multicols}{2} % Two-column layout throughout the main article text

\section{Introdução}

\lettrine[nindent=0em,lines=3]
A indústria cinematográfica é um dos setores de entretenimento mais populares e lucrativos do mundo. Segundo dados da \textit{Motion Picture Association of America} (MPAA), a receita de bilheteria nos Estados Unidos aproxima-se da cifra de US\$ $9.5$ bilhões no ano de 2006 \cite{cinema_livro}. No entanto, a produção de um curta ou longa metragem requer valores altos de investimento inicial. Um filme pertencente a uma companhia integrante da MPAA, por exemplo, custa em média US\$ $100.3$ milhões, incluindo os custos de produção, comercialização e \textit{marketing} \cite{cinema_livro}. Contudo, como toda atividade de empresa, a produção de filmes é uma atividade de risco, tendo em vista a relativa imprevisibilidade da preferência do público.

Este trabalho procura analisar os perfis de diferentes filmes, suas características e avaliações de usuários, para identificar possíveis padrões e informações que possivelmente caracterizam se um filme será um sucesso ou um fracasso. Tais informações minimizariam o risco na produção de filmes e auxiliariam diretores e produtores nesse processo, guiando-os por parâmetros que aumentariam significativamente as chances de sucesso no empreendimento.

A maioria das pesquisas nessa área busca analisar as críticas e \textit{reviews}, classificando-as como positivas ou negativas, para compreender a opinião do público quanto ao filme \cite{artigo1, artigo2, artigo3}. Outra abordagem comum são os sistemas de recomendação que tentam, através das críticas e \textit{reviews} prévias de um determinado usuário, identificar possíveis filmes recomendáveis \cite{artigo4}. Trabalhos com o foco na recepção do público aos filmes são raros e escassos.

Partindo doe dados coletados na \textit{Internet Movies Database} (IMDb), nós coletamos diferentes variáveis (gênero, duração, faixa etária, etc.) e, a partir destas e de algoritmos de \textit{machine learning}, classificá-los como "Bom", "Regular" ou "Ruim". As classificações foram definidas a partir das avaliações médias dos usuários (\textit{rating} no site. Utilizando árvores de decisão conseguimos uma média de 66\% de acerto, indicando possíveis padrões e variáveis a serem mais profundamente estudadas.

%------------------------------------------------

\section{Estado da Arte}

Conceitos Básicos.
\begin{compactitem}
\item Árvore de decisão.
\item Outros algoritmos.
\end{compactitem}
Falar dos trabalhos relacionados.

%------------------------------------------------

\section{Solução Proposta}

Solução conceitual
\begin{compactitem}
\item Instanciação.
\item Falar sobre a evolução do modelo.
\end{compactitem}

%------------------------------------------------

\section{Resultados da Avaliação}

\subsection{Prova de Conceito}

Metodologia.

Resultados.

Avaliação.

Limitações.

\subsection{Discussão}

Discussão dos resultados.

Gráficos.

\section{Conclusão}

Conclusão geral do trabalho.

%----------------------------------------------------------------------------------------
%	REFERENCE LIST
%----------------------------------------------------------------------------------------

\begin{thebibliography}{99} % Bibliography - this is intentionally simple in this template

\bibitem[Figueredo and Wolf, 2009]{Figueredo:2009dg}
Figueredo, A.~J. and Wolf, P. S.~A. (2009).
\newblock Assortative pairing and life history strategy - a cross-cultural
  study.
\newblock {\em Human Nature}, 20:317--330.
 
\end{thebibliography}

%----------------------------------------------------------------------------------------

\end{multicols}

\end{document}
