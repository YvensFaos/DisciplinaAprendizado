\documentclass[conference,a4paper]{IEEEtran}

\usepackage{cite}
\usepackage[pdftex]{graphicx}
\usepackage[cmex10]{amsmath}
\usepackage[pdftex]{graphicx}
\usepackage{array}
\usepackage{url}

\DeclareGraphicsExtensions{.pdf,.jpeg,.png}
\interdisplaylinepenalty=2500

\begin{document}

\title{Automatic movie ratings prediction using machine learning}

\author{
Mladen Marovi\'{c}, Marko Mihokovi\'{c}, Mladen Mik\v{s}a,
Sini\v{s}a Pribil, and Alan Tus\\ 
University of Zagreb, Faculty of Electrical Engineering
and Computing\\ 
Unska 3, Zagreb, Croatia\\
Email: \{mladen.marovic, marko.mihokovic, mladen.miksa,
sinisa.pribil, alan.tus\}@fer.hr\\
}

\maketitle

\begin{abstract}
Recommendation systems that model users and their interests are often used for
the improvement of various user services. Such systems are usually based on
the automatic prediction of user's ratings of the items provided by the
service. This work presents an overview of some of the methods for automatic
prediction of ratings in the domain of movie ratings. Chosen methods are based on
various approaches described in related work. While working, the methods use both
users and the items features. For the purpose of this work, data was gathered
from the publicly available movie database IMDb. Within is an implementation of
the chosen methods and their evaluation using the gathered data. The results show
an improvement in comparison to the chosen baseline methods.
\end{abstract}

\IEEEpeerreviewmaketitle



\section{Introduction}

\section{Predicting movie ratings}

Li and Yamada created a system to predict movie ratings based on decision trees
\cite{li2005movie}. Decision trees use selected film features when predicting a
movie rating. User preferences for the above mentioned film features were modeled
by building a separate decision tree for each user.

For their hybrid system, \cite{spiegel2009hydra} are proposing a model which
contains user and movie features, as well as all available ratings. Rating
prediction is then done by using \emph{singular value decomposition method}
(\emph{SVD}) and \emph{k nearest neighbors algorithm} (\emph{kNN}). Authors
call this method a \emph{SVD-kNN} method.

\section{Methods for movie ratings prediction}

\subsection{Content based methods}

One of the basic content based methods used in this work is the regression tree. Each tree
represents a user profile created using the users’ ratings. The features used in
this work are: genres, actors, directors and scenarists. All features are
represented using a binary vector, meaning that each row corresponds to a
genre, actor, director or scenarist. The rows of each vector that fit a movie
are marked with a one (1) and the rest is marked with a zero (0). When building
a regression tree the minimum square error criterion was used. Classification
trees were taken into consideration as well but preliminary testing showed that
they produced inferior results.
 

\subsection{Collaborative methods}

\emph{Personality diagnosis} method, taken from \cite{pennock2000collaborative},
models user's personality and uses it in the prediction of ratings. If there are
$m$ movies in total, personality type of user $i$ can be presented as a vector
$\mathbf{R}^\mathrm{true}_i = \langle R^\mathrm{true}_{i1}, R^\mathrm{true}_{i2},
\dots, R^\mathrm{true}_{im} \rangle$, where the ratings $R^\mathrm{true}_{ij}$
represent ``true'' ratings of the movie $j$ for the user $i$. Additionally, it is
assumed that the actual user ratings have Gaussian noise added to the ``true''
ratings. For user $i$ and movie $j$, the recorded rating presents a realization
of the random variable, governed by normal distribution, whose mean value is
equal to the ``true'' rating $R^\mathrm{true}_{ij}$:
\[
P(R_{ij} = r | R_{ij}^\mathrm{true} = r_t) \propto e^\frac{-(r - r_t) ^
2}{2 \sigma^2}.
\]
It is assumed that the acquired rating vectors form a representative distribution
of personality types present in the population. User $u$ for which the prediction
is being made can belong to any of the observed personalities $\mathbf{R}_1,
\dots, \mathbf{R}_n$ with probability $1 / n$. Knowing some of the ratings of
user $u$, the probability that he belongs to personality type $\mathbf{R}_i$ can
be calculated by:
\begin{multline*}
P(\mathbf{R}_u^\mathrm{true} = \mathbf{R}_i | R_{u1} = r_1, \dots,
R_{um} = r_m) \\
\shoveleft{\propto P(R_{u1} = r_1 | R_{u1}^\mathrm{true} = R_{i1})} \\
\cdots P(R_{um} = r_m | R_{um}^\mathrm{true} = R_{im})
P(\mathbf{R}_u^\mathrm{true} = \mathbf{R}_i).
\end{multline*}
Finally, the probability distribution of the rating for movie $i$ and current
user $u$ can be estimated using previous probabilities for personality types as
follows:
\begin{multline}
P(R_{ui} = r_i | R_{u1} = r_1, \dots, R_{um} = r_m) \\
\shoveleft{= \sum\limits_{j = 1}^{n} P(R_{ui} = r_i |
\mathbf{R}_u^\mathrm{true} = \mathbf{R}_j)} \\
\cdot P(\mathbf{R}_u^\mathrm{true} = \mathbf{R}_j | R_{u1} = r_1, \dots,
R_{um} = r_m).
\end{multline}
The rating with highest probability is then the predicted rating of user $u$ for
movie $i$. The method is defined only with the deviation of normal
distribution $\sigma$.

\emph{Probabilistic latent semantic analysis} was implemented according to
\cite{hofmann2003collaborative}. Intuitively, latent variables represent hidden
causes for ratings that user gave to each movie. Users are divided into groups
according to similar rating causes using latent variables. Formally, probability
that a user $u$ rates movie $i$ with rating $r$ can be written as:
\[
p(r |u, i) = \sum\limits_z p(r | i, z) P(z | u),
\]
where $z$ is latent variable which represents hidden causes. Probability that the
cause $z$ is responsible for the ratings of user $u$ is presented by term $P(z |
u)$, while $p(r | i, z)$ is a probability distribution for the ratings of a known
movie $i$ according to the cause $z$. It is assumed that this probability
distribution is a normal distribution. Expectation maximization algorithm is used
for the estimation of distribution's parameters  $\mu_{i,z}$, $\sigma_{i,z}$ and
probability $P(z|u)$. $E$-step of the algorithm is represented by equation:
\[
P(z | u, r, i; \hat{\theta}) = \frac{\hat{p}(r | i, z) \hat{P}(z |
u)}{\sum_{z'} \hat{p}(r | i, z') \hat{P} (z' | u)},
\]
where the hat symbol presents probabilities according to parameters
$\hat{\theta}$, that is, previous estimations for the probability's parameters.
Probabilities of latent variables $z$ for all recorded ratings are calculated
using the stated equation. This probabilities are then used in the $M$-step for
the estimation of probabilities $P(z|u)$:
\[
P(z | u) = \frac{\sum_{\langle u', r, i\rangle : u' = u} P(z | u, r, i;
\hat{\theta})}{\sum_{z'}\sum_{\langle u', r, i\rangle : u' = u} P(z' | u, r, i;
\hat{\theta})}
\]
and distribution parameters $\mu_{i,z}$ and $\sigma_{i,z}$:
\begin{align*}
\mu_{i, z} &= \frac{\sum_{\langle u, r, i' \rangle : i' = i} r\; P(z|u, r,
i; \hat{\theta})}{\sum_{\langle u, r, i' \rangle : i' = i} P(z|u, r, i;
\hat{\theta})}, \\
\sigma_{i, z} &= \frac{\sum_{\langle u, r, i' \rangle : i' = i} (r -
\mu_{i,z})^2\; P(z|u, r, i; \hat{\theta})}{\sum_{\langle u, r, i' \rangle : i' =
i} P(z|u, r, i; \hat{\theta})},
\end{align*}
where $\langle u, r, i \rangle$ represents the set of all recorded ratings. All
ratings are normalized according to the mean rating $\mu_u$ and deviation
$\sigma_u$ of the user according to equation:
\begin{equation}
(u, r, i) \mapsto (u, r', i), \quad \mbox{using}\quad r'=\frac{r -
\mu_u}{\sigma_u}.
\label{eq:norm}
\end{equation}
Using the calculated parameters, the predicted rating can be obtained as a
mathematical expectation for $p(r | u, i)$:
\[
E[r | u, i] = \mu_u + \sigma_u \sum\limits_z P(z | u) \mu_{i, z}.
\]


\subsection{Hybrid methods}

Hybrid method \emph{SVD-kNN}, described in \cite{spiegel2009hydra}, was
implemented for the purposes of this study. All available user and movie
features are placed in the matrix:
\[
H_{m \times n} = 
\left[ {\begin{array}{cc}
 R_{u \times i} & U_{u \times y} \cdot w_{2}  \\
 I_{x \times i} \cdot w_{1} & 0_{x \times y}  \\
\end{array} } \right],
\left( {\begin{array}{cc}
 m = u + x  \\
 n = i + y  \\
\end{array} } \right)
\]
where $R_{u \times i}$ is a matrix of ratings which users gave to movies they 
watched, $U_{u \times x}$ is a user feature matrix, $I_{x \times i}$ is a movie
feature matrix and $w_{1}$ and $w_{2}$ are weighted factors that are used to
define at which ratio do movie and user features affect on prediction result.
Note that filling matrix $H$ with zeros $0_{x \times y}$ is necessary to
preserve its rectangular shape. Figure~\ref{fig:hibridniModel} shows an example
of matrix $H$.

\begin{figure}
\begin{center}
%\includegraphics[width=\columnwidth]{img/hibridniModel.jpg}
\caption{Primjer matrice na kojoj se temelji \emph{SVD-kNN}.}
\label{fig:hibridniModel}
\end{center}
\end{figure}

In order to achieve better rating predictions, it is necessary to remove user
preferences, movie popularities and other global effects from matrices $R$, $U$
and $I$. For rating matrix $R$ this is done by 
%\glqq{}averaging\grqq{} 
rating
values in form of subtracting weighted combination of overall- ($\bar{r}$),
user- ($\bar{r}_{u}$) and movie-average ($\bar{r}_{i}$) from the original
rating:
\[
\tilde{r}_{ui} = r_{ui} - \alpha\bar{r} - \beta\bar{r}_{u} - \gamma\bar{r}_{i}.
\]
The parameters $\alpha$, $\beta$ and $\gamma$ determine the influence of each
effect. Feature matrices, $U$ and $I$, consist only of values $0$ or $1$, so
previously mentioned normalization method is not applicable here. Instead, the
following expression is used:
\[
\tilde{F} = M F N,
\]
where $F$ is a user- or movie-feature matrix and $M$ and $N$ are diagonal
matrices with values:
\[
M_{xx} = \frac{1}{\sqrt{\sum\limits_{n}F_{xn}}} \quad \mbox{i}\quad N_{yy} =
\frac{1}{\sqrt{\sum\limits_{m}F_{my}}}.
\]

After removing global effects in rating and feature data, \emph{SVD} method is
used on resulting matrix $H$. \emph{SVD} is an eigenvalue generalization for
non-square matrices which is often used in signal processing and statistics.
Linear decomposition procedure factorizes starting matrix into three
low-dimensional resulting matrices containing left-singular vectors ($V$),
singular values ($S$) and right-singular vectors ($W$) respectively:
\[
H_{k} = V_{m \times k} \cdot S_{k \times k} \cdot W^T_{k \times n}.
\]
Parameter $k$ defines number of extracted eigenvectors and is crucial for model
accuracy. Matrices $V$, $S$ and $W$ are used to calculate compound matrices $X
= [VS^{\frac{1}{2}}]$ and $Y = [S^{\frac{1}{2}}W^T]$ which can be considered as
separate user and movie concepts and use as a basis for collaborative rating
prediction methods. Using the k nearest neighbor algorithm, prediction is done
with expression:
\[
\hat{r}_{ui} = \frac{\sum\limits_{j \subset \mathit{ocijenjeniFilmovi}(u)}
s_{ij} r_{uj}}{\sum\limits_{j \in \mathit{ocijenjeniFilmovi}(u)}
|s_{ij}|},
\]
where $s_{ij}$ is a cosine similarity defined analogous to expression
\eqref{eqn:simCosine}. Model accuracy is also dependent on the number of
neighbor elements ($n$) that are included in the calculation.

\section{Evaluation}

All methods mentioned before are implemented and tested using a \emph{Matlab}
software package. This section describes training and test data, as well as
performance measures that are used in experiments. Last subsection contains
experiment results and their comments.

\subsection{Training and test data}

Movie and user data is collected from publicly available
IMDb\footnote{\texttt{http://www.imdb.com/}}. Movie ratings in collected data
are represented by the ratings that users gave in their reviews. Apart from the
ratings, following movie features are collected: title, genres, year of
release, directors, scenarists and actors. It is possible that some movies do
not have all of the listed features. Although bigger data set is collected, the
set of $1059$ users, $9428$ movies, $1000$ actors, $1066$ directors, $1060$
scenarists, $28$ genres and $65581$ ratings in range from $1$ to $10$ is used
for the purposes of this work.

Training data set is generated using \emph{AllButOne} method described In
%\citep{hofmann2003collaborative}. 
One rating from every user in training set is
left out and used to create a test set. Thus the training set contains $64522$
ratings, while test set contains $1059$. Average number of ratings per movie is
$7.57$, per user is $60.93$, and average rating value is $7.22$.


\subsection{Performance measures}

\subsection{Results}

The regression tree was trained using ten folded cross-validation that showed the
best places to trim the tree. The results obtained in this work are inferior to
baseline methods. Such a result can be explained with too few training samples
per user and a very sparse distribution of features per film. A small number of
sparse vectors cause problems when discerning a movie.
 

\section{Conclusion}

Predicting user ratings for some object presents an interesting and well formed
problem. Many different approaches to solving this problem were used with
differences in used methods and given results. This works gives a comparison of a
number of methods used in automatic prediction of movie ratings. The used dataset
was collected from the publicly available IMDb. While testing, the probabilistic
latent semantic analysis proved as best.

Future plans include more detailed testing with different datasets because the
currently used dataset does not contain a required diversity in user ratings.
Some of the implemented methods were not tested thoroughly enough because of
their computational complexity and should be tested in detail. Other methods suggested
in related work should be taken into consideration as well. Finally, a
combination of classifiers could be tested and the use of weighed voting when
predicting.
 

\nocite{*}
\bibliographystyle{IEEEtranS}
\bibliography{IEEEabrv,su}

\end{document}

